% BEGINNING OF COMMENT
%
% This is a section for you to write your introduction to your personal letter. Please note that you need to add two backslashes // to properly make the indentation work. I could solve it with code but can't be bothered.
%
% END OF COMMENT

\documentclass[../main.tex]{subfiles}
\begin{document}
\par
I augusti 2020 flyttade jag till Stockholm för att studera datateknik på KTH. Under det första året av mina studier arbetade jag deltid som frilansare med att hjälpa kunder att modernisera sina varumärken. Jag gjorde detta främst genom webbdesign, logotyper och marknadsföring.
\\

Under sommaren och hösten 2021 fick jag en heltidspraktik för AWS som underhåller deras servrar i Eskilstuna. Det har gjort det möjligt för mig att växa och lära mig mer om serverhårdvara, nätverk, virtualisering, operativsystem och molnindustrin.
\\

Denna praktik gav mig intresset och kunskapen som jag behövde för att börja experimentera med Linux och servermjukvara på min fritid. Efter att ha gått ner i ett ”rabbit-hole" av "homelab"-teknologier som Proxmox, Networking, Raspberry Pis, etc; fick jag den erfarenhet jag behövde för att vara värd för mina egna tjänster. Från och med idag är jag själv värd för min egen router, git-server, mediaserver, lösenordshanterare, molnlagring (nextcloud), omvänd proxy (nginx) och spelservrar. Allt detta på tre maskiner för att bygga ett högtillgänglighetskluster för att säkerhetskopiera all min data, som allt är dokumenterat på "homelab" github-repository.
\\

Utöver detta arbetar jag ofta på min egen hemsida, calexanderberg.com, på min fritid. Jag skrev nyligen om den i vanlig HTML eftersom jag tyckte det var svårt att hela tiden uppdatera mina webbutvecklingswebbplatser pga informationen blev irrelevant eller om jag tyckte att designen var tråkig. Jag har skrivit min webbplats i flera ramverk tex, React, Vue, Rust/Webassembly, Html/Css, etc; jag tyckte mest att detta var för tidskrävande. Jag har även en blogg som är skriven i Zola som jag använder för att skriva om teknik och filosofi, länken till denna finns på min hemsida.
\\

Sedan min praktik arbetade jag en kort tid som elektronisk systemsupportagent på SEB innan jag arbetade som QA-testanalytiker på Nasdaq. Jag anställdes för att hjälpa till att skriva och skapa automatiseringsskript för Nasdaqs interna system på de europeiska marknaderna. Detta gjorde att jag kunde arbeta med teknologier som Pandas, Playwright och Robot framework, bara för att nämna några. Det här jobbet har gett mig mycket mer erfarenhet av Git, GitLab, CI/CL, WSL och överlag hur stora projekt hanteras mellan flera teammedlemmar. Vi använder också Agile och scrum i vårt arbetsflöde för att hjälpa oss hantera problem under längre tidsperioder. Jag lämnade Nasdaq i slutet av 2023 för att kunna fokusera på att avsluta mina studier.
\\

\end{document}
